%%%%%%%%%%%%%%%%%%%%%%%%%%%%%%%%%%%%%%%%%
% Long Lined Cover Letter
% LaTeX Template
% Version 1.0 (1/6/13)
%
% This template has been downloaded from:
% http://www.LaTeXTemplates.com
%
% Original author:
% Matthew J. Miller
% http://www.matthewjmiller.net/howtos/customized-cover-letter-scripts/
%
% License:
% CC BY-NC-SA 3.0 (http://creativecommons.org/licenses/by-nc-sa/3.0/)
%
%%%%%%%%%%%%%%%%%%%%%%%%%%%%%%%%%%%%%%%%%

%----------------------------------------------------------------------------------------
%	PACKAGES AND OTHER DOCUMENT CONFIGURATIONS
%----------------------------------------------------------------------------------------

\documentclass[10pt,stdletter,dateno,sigleft,a4paper]{newlfm} % Extra options: 'sigleft' for a left-aligned signature, 'stdletternofrom' to remove the from address, 'letterpaper' for US letter paper - consult the newlfm class manual for more options

\usepackage{charter} % Use the Charter font for the document text
\usepackage{etoolbox}
\usepackage{url}
\makeatletter
\patchcmd{\@zfancyhead}{\fancy@reset}{\f@nch@reset}{}{}
\patchcmd{\@set@em@up}{\f@ncyolh}{\f@nch@olh}{}{}
\patchcmd{\@set@em@up}{\f@ncyolh}{\f@nch@olh}{}{}
\patchcmd{\@set@em@up}{\f@ncyorh}{\f@nch@orh}{}{}
\makeatother

\textheightsize{750pt}
%\textwidthsize{501pt}
%\leftmarginsize{56pt}

\newsavebox{\Luiuc}\sbox{\Luiuc}{\parbox[b]{1.75in}{\vspace{0.5in}
\includegraphics[width=2\linewidth]{logo_all.png}}} % Company/institution logo at the top left of the page
\makeletterhead{Uiuc}{\Lheader{\usebox{\Luiuc}}}

\newlfmP{sigsize=10pt} % Slightly decrease the height of the signature field
\newlfmP{addrfromphone} % Print a phone number under the sender's address
\newlfmP{addrfromemail} % Print an email address under the sender's address
\PhrPhone{Phone} % Customize the "Telephone" text
\PhrEmail{Email} % Customize the "E-mail" text

\lthUiuc % Print the company/institution logo

%----------------------------------------------------------------------------------------
%	YOUR NAME AND CONTACT INFORMATION
%----------------------------------------------------------------------------------------

\namefrom{Andrew R. McCluskey \& Benjamin J. Morgan} % Name

\addrfrom{
\today\\[12pt] % Date
Department of Chemistry \\ % Address
University of Bath \\
Claverton Down,
Bath \\ UK,
BA2 7AY
}

\phonefrom{(+44) 1225 38 6893} % Phone number

\emailfrom{a.r.mccluskey@bath.ac.uk} % Email address

%----------------------------------------------------------------------------------------
%	ADDRESSEE AND GREETING/CLOSING
%----------------------------------------------------------------------------------------

\greetto{Dear Editor,} % Greeting text
\closeline{Yours sincerely,} % Closing text

\nameto{} % Addressee of the letter above the to address

\addrto{Journal of Applied Crystallography\\ International Union of Crystallography}

%----------------------------------------------------------------------------------------

\begin{document}
\begin{newlfm}

%----------------------------------------------------------------------------------------
%	LETTER CONTENT
%----------------------------------------------------------------------------------------

Please kindly consider the enclosed manuscript ``Introducing classical molecular dynamics simulation to users of scattering'' for inclusion as a Teaching and Education article in the Journal of Applied Crystallography.

Classical molecular dynamics simulations are often applied in the analysis of experimental data, particularly in small angle scattering and diffraction [Pan \emph{et al.}, BBA - Biomembranes, \textbf{2012}, 1818, 2135; Hub, Curr. Opin. Struct. Biol., \textbf{2018}, 49, 18; Perkins \emph{et al.}, J. Appl. Crystallogr., \textbf{2016}, 49, 1861].
However, these simulation techniques often require specific training that is not widely accessible, usually being offered as a component of a limited (both in numbers and by location) training course.
In order to reduce the barrier to entry for those interested in using classical molecular simulation to assist in the analysis of scattering data, we have developed the ``The interaction between simulation and scattering'' open educational resource as a component of the pythoninchemistry project.

This open educational resource builds on the success of the pylj classical simulation software [McCluskey \emph{et al.}, J. Open Source Educ., \textbf{2018}, 1, 19] and the IUCr Journals Award winning presentation at the SAS2018 conference [McCluskey, DOI:10.6084/m9.figshare.7150820] to create an interactive learning resource to introduce users of experimental techniques to classical simulation and molecular dynamics, before finally introducing how these methods may interact with scattering techniques.
The included manuscript details the design and use of the open educational resource, while the resource itself can be found at \url{pythoninchemistry.org/sim_and_scat}.
We believe that this educational resource offers a significant benefit to the scattering and diffraction community by making a straightforward guide available under an open, permissive license.

In conclusion, we believe that our manuscript is suitable for publication as a Teaching and Education article in the Journal of Applied Crystallography and welcome any feedback you might have.

%----------------------------------------------------------------------------------------

\end{newlfm}
\end{document}
